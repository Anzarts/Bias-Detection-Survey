% This is samplepaper.tex, a sample chapter demonstrating the
% LLNCS macro package for Springer Computer Science proceedings;
% Version 2.21 of 2022/01/12
%
\documentclass[runningheads]{llncs}
%
\usepackage[T1]{fontenc}
% T1 fonts will be used to generate the final print and online PDFs,
% so please use T1 fonts in your manuscript whenever possible.
% Other font encondings may result in incorrect characters.
%
\usepackage{graphicx}
% Used for displaying a sample figure. If possible, figure files should
% be included in EPS format.
%
% If you use the hyperref package, please uncomment the following two lines
% to display URLs in blue roman font according to Springer's eBook style:
%\usepackage{color}
%\renewcommand\UrlFont{\color{blue}\rmfamily}
%\urlstyle{rm}

\usepackage{algorithm}
\usepackage{algpseudocode}
\usepackage{amsmath}
\usepackage{amssymb}
\usepackage{amsthm}
\usepackage{xstring}
\usepackage{color}
\usepackage{comment}
\usepackage{paracol}
\usepackage{multirow}
\usepackage{csquotes}
\usepackage{paracol}
\usepackage{hyperref}
%\definecolor{refcolor}{rgb}{0, 0, 0.5}
\hypersetup{colorlinks=true, citecolor=black, linkcolor=black, urlcolor=black}

\globalcounter{figure}
\globalcounter{table}
\globalcounter{section}
\globalcounter{subsection}
\globalcounter{subsubsection}
\globalcounter{footnote}

\DeclareMathOperator*{\argmax}{arg\,max}
\DeclareMathOperator*{\argmin}{arg\,min}
\DeclareMathOperator*{\mean}{mean}
\DeclareMathOperator*{\sd}{sd}

%
\begin{document}
%
\title{Paper Review}
%
%\titlerunning{Abbreviated paper title}
% If the paper title is too long for the running head, you can set
% an abbreviated paper title here


\author{Benjamín Farías Riquelme\inst{1}}%\orcidID{0000-1111-2222-3333}}

\authorrunning{B. Farías Riquelme}
% First names are abbreviated in the running head.
% If there are more than two authors, 'et al.' is used.

\institute{Universidad de Chile\\
\email{benjamin.farias@ing.uchile.cl}}


\maketitle % typeset the header of the contribution

\begin{abstract}
%The abstract should briefly summarize the contents of the paper in 150--250 words.

[...]

%\keywords{Natural Language Processing  \and Counterfactual Explanation \and NLP interpretability}
\end{abstract}
%
%
%

% Debe contener:
%
% - Intro
% - Revisión sistemática de 10-20 artículos
% - Discución y análisis
%
%

\section{Introduction}
\label{sec:intro}

% As the influence and scope of algorithms increase, academics, policymakers and journalists have raised concerns that these tools might inadvertedly encode and entech human biases \cite{measures-of-fairness} *

% predictive models in NLP are sensitive to (often unintended) bias \cite{predictive-bias-NLP} *

% bias can lead to social undesirable effects, by systematically mispredicting or underserving certain demographic goups \cite{predictive-bias-NLP}

% undetected and unassessed biases can lead to negative consequences \cite{predictive-bias-NLP} *

% Biased predictions emerge from hidden on neglected biases in the model or data \cite{bias-and-fairness-in-ML} *

% when applied in real-world applications, biased models have the potential for negative societal impacts \cite{sociodemographic-bias,NLP-risk-child-protective-service,critical-survey-on-bias} *

Alongside the increase in capacity and influence of algorithms, there is an increase in the concerns and risks of them inadvertedly perpetuating human biases~\cite{measures-of-fairness,Weapons-of-math-destruction}. This phenomenon is particularly evident in the context of neural networks developed for Natural Language Processing (NLP), as they learn directly from human-generated texts. The delegation of the decision-making process to these algorithms has the potential to engender negative societal impact if there are undetected or neglected biases~\cite{critical-survey-on-bias,NLP-risk-child-protective-service,sociodemographic-bias,bias-and-fairness-in-ML,predictive-bias-NLP}. We will use the term predictive model (in NLP), to refer to any model that takes text as input, and produce a prediction, decision or classification as output (e.g. toxic text classification or spam filtering). To ilustrate the risks of unassessed biases in predictive models in NLP, consider the following toy example.

\begin{displayquote}
A certain company is acused of favoring men over women in their hiring process. To solve this problem, the company decides to leave the process of a neural network, which determines if an applicant should be hired or not, based on their anonimized resume. The network is trained with the data of the previous processes, so it replicates the same biases. The next time the company is questioned for their hiring process, they respond that it is managed by an algorithm and, unlike humans, algorithms are objective, so if it more men than women are hired, it must be because there are more men better qualified for the job.
\end{displayquote}

% ----------------------------

% Mayor parte de los trabajos en sesgo se centran en medir o reducir un sesgo determinado. Ambos casos estan interconectados -> cuando se habla de des-sesgar, en realidad es reducir una métrica en particular

% El objetivo de esta survey no es decir que harmful o que no, sino sólo categorizar sesgos en general y cómo encontrarlos

In the example, the bias is the association of gender with qualification for the job, and it is originated by the use of biased training data. In Section \ref{sec:bias} we provide a more precise definition of bias in NLP and its origins.

A substantial corpus of research has been dedicated to the examination of prediction bias in NLP, and biases in NLP in general. Most of them can be encapsuled in three categories: characterization of bias and its risks, measuring bias, and debiasing. The last two categories are closely intertwined, given that debiasing methods, which aim to remove a specific bias, often do so by reducing the metrics defined in the bias measuring literature. However, it has been observed that the metric reduction approach is more likely an elimination of symptoms rather than biases~\cite{fairness-survey,intrinsic-not-correlate,lipstick}.

In general, bias measuring works choose a social bias, such as gender or racial bias, and propose a metric to quantify the presence of this bias in either the model representations or responses. These are referred in the literature as intrinsic and extrinsic metrics, respectively~\cite{quantifying-social-bias,sociodemographic-bias}. This concept is closely related to that of fairness metrics~\cite{measures-of-fairness}, which, in turn, are metric directly designed to measure the consequences and symptoms of biases (or other problematic elements), rather than the bias itself.

% ----------------------------

% we propose to address the problem from a new scope | give more weight to this other scope

% No son todos los trabajos en bias, pero se espera que sean los suficientemente relevantes y representativos

% No todos son directamente de bias, pero son metodologias que se pueden utilizar para el caso

% Decidir si un bias es harmful va despues de definir el bias -> se deben primero encontrar los sesgos, deseados o no

% Denuevo, el objetivo es discovery, no measuring or mitigation

% Aquí bias es bias y unintended bias es unintended bias

% Diferenciar bias de fairness

% Otra categoría útil puedeser sí es que el método sólo advierte de la presencia de sesgo o da algún indicio de su naturaleza -> métricas de fairness tenderían al primer caso

% poner running examples en la intro, usar retórica de fairness para justifica interés en sesgos

% bias discovery no es sólo señalar si existe sesgo o no, sino entregar algún indicio de caracterización del sesgo en cuestión

We propose an alternative approach to addressing the issue of bias in predictive models. Rather than measuring the extent of bias in a model, we endeavor to identify the biased associations that the model makes. To better explain this idea, we will continue with the hiring model example.

\begin{displayquote}
An association against gender bias, dertermined to demonstrate that the model used by the company is biased, recolect of 100 applicants and their results. They find that men were accepted 4 times more than women, even when they have similar background. With this fairness test, the association is sure that the model favors men over women, however, the company points out a key detail, the resumes are anonimized, so the model does not has any information regarding gender, and thus cannot be biased. The association review the applications again, this time focusing on the diference between accepted and rejected resumes. They find that the rejected resumes tend to use longer sentences with more unique words, and that this characteristic is more frequent in women resumes\footnote{This toy example is based on the findings of Qu et al.\cite{gender-resume-differences}}. After editing the resumes to have a similar style and passing them to the model, it is found that the rejection rate is now equitative between men and women, probing that it was, in fact, biased.
\end{displayquote}

Here, even if a metric indicates that the answers of the model are biased, it does not explain why it happens. After examining a set of input-output samples, it is found a correlation between a single attribute of the input and a particular output, an association can be called a bias. We denominate this procedure of finding associations as bias detection. In the example, the association is identified through manual examination. However, it would be preferable to have a mechanism capable of automatically detecting bias. 

In this short survey we review some methods that can be eployed, or repurposed, to perform this task. The scope of the survey is restricted to methods that can be applied on predictive models with transformer architecture. We divide the methods in two categories: examination methods (Section \ref{sec:examination}), that examinates the representations used or generated by the model and its operations, and example generation methods (Section \ref{sec:examples}), that generates examples that might show the biases of the model. We also indicate if the methods are useful for a confirmatory or exploratory bias analysis. The definitions of both analyses are provided in Section \ref{sec:indirect-effect}.

\begin{table}
    \centering
    \resizebox{\columnwidth}{!}{
    \begin{tabular}{|l|cc|cc|}
        \hline
         & \multicolumn{1}{c|}{Examination} & \multicolumn{1}{c|}{Example Generation} & \multicolumn{1}{c|}{Confirmatory} & \multicolumn{1}{c|}{Exploratory} \\ \hline
        WEAT & $\checkmark$ &  & $\checkmark$ &  \\ \cline{1-1}
        SEAT & $\checkmark$ &  & $\checkmark$ &  \\ \cline{1-1}
        CEAT & $\checkmark$ &  & $\checkmark$ &  \\ \cline{1-1}
        RIPA & $\checkmark$ &  & $\checkmark$ &  \\ \cline{1-1}
        AG & $\checkmark$ &  &  & $\checkmark$ \\ \cline{1-1}
        MiCE &  & $\checkmark$ &  & $\checkmark$ \\ \cline{1-1}
        GYC &  & $\checkmark$ &  & $\checkmark$ \\ \cline{1-1}
        POLYJUICE &  & $\checkmark$ &  &  \\ \cline{1-1}
        PIS &  & $\checkmark$ &  & $\checkmark$ \\ \cline{1-1}
        MEIO &  & $\checkmark$ &  & $\checkmark$ \\ \cline{1-1}
        R-LACE & $\checkmark$ &  & $\checkmark$ & $\checkmark$ \\ \cline{1-1}
        TEA & $\checkmark$ &  & $\checkmark$ & $\checkmark$ \\ \hline
    \end{tabular}}
    \caption{Caption.}
    \label{tab:C_mlm}
\end{table}

% address limitations of the survey at some point

% Unintended bias es más dificil de definir que bias en general -> requiere métodos más epecíficos para encontrarlo -> es más fácil pasar por alto otros sesgos

% Selección de papers es, en primer lugar, por su posible aplicación en búsqueda de sesgos, pero también por su relevancia "historica" en el campo de sesgos en NLP

% Dividir los métodos según la información que requieren para buscar los sesgos | exloratory - verification


%\section{\textcolor{red}{Review}}
\label{sec:review}

\begin{itemize}
    \item \cite{counterfactual-explanations} defines counterfactual explainers.
    \item \cite{NLP-counterfactual-survey} gives a taxonomy for NLP counterfactuals.
    \item \cite{GYC} GYC.
    \item \cite{MiCE} MiCE.
    \item \cite{T-PGD} T-PGD.
    \item \cite{synthetisizing-Nguyen,synthetisizing-Barbalau} preferred input synthesis.
    \item \cite{counterfactual-fairness} counterfactual fairness.
    \item \cite{counterfactual-fairness-text-clf} counterfactual fairness in token classification.
    \item \cite{Causal-mediation-survey,Indirect-effect} causal mediation analysis.
    \item \cite{Mechanistic-survey} Mechanistic interpretability survey.
    \item \cite{Adversarial-attack-neuron-activation} Adversarial attacks on the interpretation of neuron activation maximization.
\end{itemize}

\subsection{Data-Centered Aproach}
\label{sec:review:data_centered}



\subsubsection{Preferred Input Synthesis}
\label{sec:review:data_centered:synthesis}



\subsubsection{Counterfactual Explanations}
\label{sec:review:data_centered:counterfactual}



\subsection{Component-Centered Aproach}
\label{sec:review:Component_centered}

\subsection{GYC}

Aims to fulfill these counterfactual's properties:

\begin{itemize}
    \item \textbf{Plausibility:} ensures that the examples are something that could occur.
    \item \textbf{Diversity:} ensure high coverage of the input space.
    \item \textbf{Goal-orientedness:} ensures that the examples deviate from the original sample on a particular aspect.
    \item \textbf{Effectivenes:} ensure the examples are useful for finding test-failures.
\end{itemize}

The proposed counterfactual explainer can direct the sample generation towards any user-specific contdition. GYC do not requires to train or fine-tune a model.

transformer-based architectures can generate a token $y_t$ conditioned on the past tokens $y_{<t}=\{y_i\}^{t-1}_{i=0}$ as follows

\begin{equation}
    \begin{split}
        o_{t}, H_{t} &= \text{LM}(y_{t-1}, H_{t-1})\\
        y_t &\sim \text{Categorical}(o_t)
    \end{split}
\end{equation}

Where $H_{t-1}$ is the history matrix that captures the dependency of $y_{t-1}$ on past tokens and $o_t$ are the logits to sample $y_t$ from a categorical distribution.

GYC generates $K$ counterfactual texts via controlled text generation, from a given text and condition, that delimitates the scope of the counterfactuals.

\begin{equation}
    \{\tilde{y}_i\}^K_{i=1} \sim P(\tilde{y}|X, \text{condition})
\end{equation}

This is achieved by perturbating the history matrix two times, first to create $\tilde{H}_t$ which enforces the reconstruction of $X$, and then to create $\hat{H}_t$ which enforces the condition. To learn the perturbations a linear combination of three loss functions is employed, one for reconstruction and one enforce the condition, plus another one to ensure diversity. The reconstruction loss maximizes the log probability of the input text, the condition loss maximizes a score assosiated to the condition and the diversity loss maximizes the entropy of the generated logits.

\section{Background}
\label{sec:background}

In this section we review some basic concepts necessary for a better discussion of Prediction Bias Analysis. In Section \ref{sec:backgroun:bias} we expand the definition of Prediction Bias with the concept of unintended bias, which is the subcategory of potentially harmful biases. %In Section \ref{sec:backgroun:indirect-effect} we review the concepts of direct and indirect effect from the causal infenrence literature, which quantify the extent to which a variable if influenced by another. In the context of Prediction Bias Analysis, they correspond to two complementary assessments, necessary to identify Prediction Bias.

\subsection{Prediction Bias in NLP}
\label{sec:backgroun:bias}

Despite of the large number of works addressing bias in NLP, there is a lack of consensus regarding the definition of bias~\cite{critical-survey-on-bias}. The discussion over the various definitions of bias constitutes a complex subject that will not be thoroughly addressed in this survey. In this work, we will restrict our analysis to the definitions outlined in Section \ref{sec:intro:social_and_prediction_bias}, where Social Bias corresponds to a bias present in human prejudices and Prediction Bias to a bias present in the model's computations. Prediction Bias in NLP has been described previous works as the prior that informs a model's predictions~\cite{fairness-survey,predictive-bias-NLP}. In essence, Prediction Bias can be understood as the associations between features of the input and the output of the model. According to this definition, all models have Prediction Bias, however, it is not something inherently ploblematic, but another gear in the model's mechanism.

Biases can be harmful when they come from harmful precedents~\cite{WEAT}. Biases that are not aligned with reality, or are aligned with a reality that we do not wish the model to learn from, are denominated as unintended biases~\cite{fairness-survey,predictive-bias-NLP}. This would be the case for the association between qualification for the job and a feature that correlates with gender in the hiring model example.

The majority of predictive models in NLP are trained with real-world text samples, which are unavoidably biased by the context in which they are written and the demographic of who writes them~\cite{geo-lexical-variation,social-media-language,secret-life-of-pronouns}. In consequence, there is a high chance that the models replicate those biases. This can lead models to pick up patterns that do not generalize to other contexts or demographics, or rely on undesired relations, resulting in unfair or harmful predictions~\cite{machine-bias-book,bias-and-fairness-in-ML,Weapons-of-math-destruction,predictive-bias-NLP}. Even if the data does not present undesired biases, models themselves may still manifest unintended biased behavior due to certain design choices~\cite{bias-and-fairness-in-ML} or by inheriting them from biased representations~\cite{man-is-to-computer,WEAT}.

Unintended biases, for predictive model in NLP, can be divided into four categories according to the source of the bias~\cite{predictive-bias-NLP}:

\begin{itemize}
    \item \textbf{Semantic Bias:} Emerges when the word-embeddings used by the model encode biased relations. \textit{The information represented in the embeddings include a Social Bias (e.g., gender in genderless words).}
    \item \textbf{Label Bias:} Emerges when the model learns predictions that diverge substantially from the ideal distribution, product of labels aligned with a (not desired) biased reality. \textit{The training data is affected by a Social Bias, as in in the hiring model example (}\textsl{Ex1}\textit{)}.
    \item \textbf{Selection Bias:} Emerges when the model learns from data that is non-representative of the distribution to where it would be applied. \textit{The training data is selection is affected by a Social Bias (e.g., use mostly texts written by middle-
aged white men).}
    \item \textbf{Overamplification:} Emerges when the model itself pick up small difference in the data, and amplify them to be much larger in the predicted outcomes. \textit{The model develops its own unintended Prediction Bias, without the influence of a Social Bias.}
\end{itemize}

As indicated in the text in cursive, these categories of origin of unintended bias correspond to different forms of how Social Bias can cause Prediction Bias. The identification of the origin of bias can help in the development of unbiased models. Nevertheless, it requieres analysis beyond the boundaries of Prediction Bias Analysis, so it will not be addressed in this work.


\section{Indirect Effect}
\label{sec:indirect-effect}

In our hiring model example (Section \ref{sec:intro}), the model express a gender bias, but the actual association encoded in the model is between the length of sentences and variety of words, which correlates with gender, and the acceptability for the job. In fact, the model does not even recieve any information regardign gender. In this toy scenario, gender has no direct effect, but it has a significant indirect effect.

The effect that a variable has over an outcome can be decomposed as the sum of its direct effect and its indirect effect. The direct effect is a quantification of the influence that a variable has on an outcome, that is not mediated by other variables. Let $X$ be the varieble whose effect we seek to assess, $Y$ the response variable, and $Z$ the set of all the intermediate variables between $X$ and $Y$. The direct effect of $X$ over $Y$ measures the sesivity of $Y$ to changes in $X$, while $Z$ is held fixed~\cite{indirect-effect}. Formaly, the direct effect of an event $X=x$ is given by

\begin{equation}
    DE(x, x^*;Y)=Y_{xZ_{x^*}} - Y_{x^*}
\end{equation}

where $x^*$ is a reference value for $X$ and the notation $Y_x$ is used to represent the value $Y$ would attain when $X$ is set to be $x$. In contrast, the indirect effect of an event $X=x$ quantify the sensitivity of $Y$, to changes in the mediators $Z$ induced by $X=x$. This is equivalent to measure the change in $Y$ when $Z$ is set to the value it would attain under $X=x$, while $X$ is held fixed at the reference value~\cite{indirect-effect}. The indirect effect is given by

\begin{equation}
    IE(x, x^*;Y)=Y_{x^*Z_{x}} - Y_{x^*}
\end{equation}

The indirect effect is an important factor to consider when analysing bias that is often neglected. In the hiring model example, the idea that the model a bias realated to $X$ (gender) is rejected at first, because is obvious that $X$ has no direct on $Y$ (the output). The inderect effect, and subsequently the bias, is probed after finding, by chance, the mediator $Z$ (length of sentences and variety of words) between $X$ and $Y$. In this case is evident that $X$ has no direct effect, as it is not part of the input, but there can be similar escenarios where $X$ is in the input. Consider the following alternative version of the example:

\begin{displayquote}
    The company, aware that their model will be acused of perpetuating gender bias, include the gender of the applicants, adding two identical copies of each of them, varying only the gender, to the training data, in other to prevent the bias. After puting the model in use, the company is denounced by the association, allegating that the model perpetuates gender bias.
\end{displayquote}

In this alternative version of the example, $X$ is modified in the input, but $Z$ remains in its original value, so the model learns the same association between $Z$ and $Y$. The direct effect of $X$ on $Y$ is low, but the bias persist. In order to achieve a complete bias detection, it is imperative to ensure that the detection method is assessing the total effect and not only the direct effect.

If, as in the example, bias is dominated by an indirect effect, it may be necessary a characterization of the mediator variables $Z$ to effectively debias the model. However, in general, $Z$ is an unknown variable, and can correspond to a feature that is not explicity represented in the data. While $X$ can be directly observed and intervented, $Z$ must be found. We denominate as confirmatory analysis to verify or measure the effect of $X$ over $Y$, and as exploratory analysis to to verify or measure the effect of a mediator $Z$, between $X$ and $Y$. In other words, confirmatory detection methods detect an association that involves a target variable, and exploratory detection methods can detect an association that involves a previously unknown variable.


\section{Examination Methods}
\label{sec:examination}

% Mechanism porque examina los mecanismos con que opera el modelo, o cómo son codificadas las palabras -> no confundir con mechanistic interpretability

\subsection{Embedding Association}
\label{sec:examination:embeddings}

This category encompas methods that can be employed to detect biases in the word representations used by the model. These can be either the word-embeddings used in the input, or the contextualized embeddings generated by the model.

\subsubsection{Word-Embedding Association Test}

% RIPA \cite{RIPA} hace lo mismo que WEAT, pero con inner product en vez de similitud coseno

% como WEAT sólo estudia relaciones entre los embeddings y no con la predicción de un modelo, no hay análisis causal
% en alguna survey hay una cita de que WEAT no correlaciona con los sesgos del modelo
% es sensible a malas definiciones de los sets -> fallar o llebar a malas conclusiones

The Word-Embedding Association Test (WEAT)~\cite{WEAT} is one of the most influential work addressing bias encoded in word-embeddings. WEAT is an adaptation of the Intrinsic-Association Test used in social psychology, to measure stereoty-related bias in word-embeddings. Given 2 sets $X, Y$ of target words (e.g. professions) and 2 sets $A, B$ of attribues words (e.g. gender nouns), WEAT provides a metric (equations \ref{eq:WEAT1} and \ref{eq:WEAT2}) that measures the differential association of the two sets of target words $X, Y$ with the attributes $A, B$, where the association between words is defined as the cosine similarity.

\begin{equation}
\label{eq:WEAT1}
    \text{WEAT}(A,B,X,Y) = \frac{\mean_{x\in X}s(x,A,B)-\mean_{y\in Y}s(y,A,B)}{\sd_{w\in X\cup Y}s(w,A,B)}
\end{equation}

\begin{equation}
\label{eq:WEAT2}
    s(w,A,B) = \mean_{a\in A}\text{cossim}(w,a) - \mean_{b\in B}\text{cossim}(w,B)
\end{equation}

WEAT was designed to be applied in contexts where the words of $X$ and $Y$ should be equaly associated to the words of both $A$ and $B$. If, for example, it is found $A$ is more associated with $X$ than $Y$, it said that the embeddings are biased.

Many works have adapted WEAT to work in other escenarios. For instance, SEAT~\cite{SEAT} measures association in sentences, replacing the word-embeddings by sentence-embeddings, and CEAT~\cite{CEAT} measures association in contextualized embedding, by computing WEAT $N$ times with random contextualized embeddings, from different sentences containing words from the target and attribute sets, and then analizing the resulting distribution. There are also alternative formulations of WEAT for the same context, such as RIPA~\cite{RIPA}, that propose to use the inner product instead of cosine similarity to measure association.

WEAT-based methods can be employed for detecting bias, by defining a threshold for the bias metric delimit from which point the embeddings are considered to be biased. However these methods are constricted by the requierement of defining the target and attribute sets. WEAT can only look for pre-determined associations, and is subseptible to error if word sets are not well defined.

A possible solution to the limitation of the target or attribute sets, could be to develope an algorithm that authomatically generates these sets, similiar to how the Intersectional Bias Detection method iterate~\cite{CEAT}, proposed by the authors of CEAT, iterate over different combinations of subsets of the attributes to find biases associated to individuals that are in the intersection of two groups. 

% CEAT Also developed de Intersectional Bias Detection (IBD) method, to automatically identify biases without rellying on pre-defined sets
% Intersectional refers to individuals that are in the intersection of two groups
% just run wefat over different attributes and words, and select the words with values over a threshold

% --------------------------------

\subsubsection{Analogy Generation}

% man is to computer etc \cite{man-is-to-computer}
% analogy generator: given words a, b, search a pair x, y that fit in the analogy a is to x as b is to y
% filter by the metric: cossim(a-b,x-y) if ||x-y|| < d else 0

% find gender sub-space by applying PCA to she - he like vectors -> gender pair difference
% then test if words are similars to this vector

One interesting feature of word-embeddings is that they have been found able to express words relation through vector difference~\cite{embedding-relations-1,embedding-relations-2}. For example, the different between the embeddings for man and woman is similar to the difference between the embeddings for king and queen. This can be expressed in an analogy of the form ``man is woman as king is to queen''. Given a pair of words $x$ and $y$, the method of analogy generation~\cite{man-is-to-computer} consist in looking for pairs $(a,b)$ that might fit in the analogy ``$x$ is to $y$ as $a$ is to $b$''. To do this, each pair $(a,b)$ is assigned a score defined by:

\begin{equation}
    S_{x,y}(a,b) = \begin{cases} \text{cossim}(x-y,a-b) & \text{if   } \|a-b\| \leq \delta \\ 0 & \text{if   } \|a-b\| > \delta \end{cases}
\end{equation}

where $\delta$ is a threshold for the distance between $a$ and $y$. Analogy generation can be employed for bias detection at the level of the word-embeddings, but have similar issues to the ones of WEAT, as it requires to pre-define a set of candidate words for $a$ and $b$, which can be under-representative.

\subsection{Concept Erasure}
\label{sec:examination:erasure}

\subsubsection{R-LACE}

% puede encontrar sesgos al comparar instancias similares en la proyección y distintas en el original

Relaxed Linear Adversal Concept Erasure (R-LACE)~\cite{R-LACE} is method designed for the task of concept erasure in the representations. That is, given a set of vector representations $X=\{x_i\}^N_{i=1}\subseteq \mathbb{R}^d$ (e.g. word-embeddings) and a set of response variables $Y=\{y_i\}^N_{i=1}$ that indicates a concept in the vectors (e.g. gender), implement some function $r: \mathbb{R}^d\to \mathbb{R}^{d'}$, such that the resulting vectors $r(x_i)$ preserve as much information as possible, while not being predictive of concept $Y$.

To erase a concept, R-LACE finds a subspace $B\subseteq \mathbb{R}^d$ that contains the information of the target concept, within the representations, and project the vector representations to the ortogonal complement of $B$. The subspace $B$ is determined by solving the minmax problem:

\begin{equation}
    \min_{\theta}\max_{P}\sum^N_{i=1}\mathcal{L}(y_i,g^{-1}(\theta^T P x_i))
\end{equation}

where $f_\theta(x) = g^{-1}(\theta^T x)$ is a generalized linear model, with parameters $\theta$ and link function $g$, $\mathcal{L}$ is a loss function, and $P$ is a $d\times d$ ortogonal projection matrix that neutralizes a rank $k$ subspace, with $k$ being an hyper-parameter of the algorithm.

Note that the definition of R-LACE does not require to have an explicit definition of the concept to be erased, just to know the response variables. R-LACE can be repurposed for bias detection, at the level of the inner representations, by erasing the concept that determines a particular prediction. R-LACE can be expanded for non-linear subspaces by applying a kernel on $f_\theta$~\cite{kernelized-concept-erasure}.

Let $X$ be the inner representations given by some inner layer of a predictive model, $Y$ a response variable that indicates if a representation $x_i$ is assigned to particular prediction or not by the model, and $r(X) = \{r(x_i)\}^N_{i=1}$ the resulting vectors after applying R-LACE on $X$ to erase $Y$. If there are two instances $a$ and $b$, such that their representations are similar after applying R-LACE, $r(x_a)\approx r(x_b)$, but not before, $x_a \not\approx x_b$, then the difference between $a$ and $b$ might show the prediction bias.

\subsection{Model Ablation}
\label{sec:examination:ablation}

\subsubsection{Targeted Edge Ablation}

Targeted Edge Ablation (TEA)~\cite{circuit-breaking} is a technique designed to remove an especific behavior of the model, by ablating a small number of edges, or pathways, between its components. In this context, given a model $M$ and a loss function $\mathcal{L}$ (that is not necessarily the one used to train $M$), a behavior is specified as a set of inputs $\mathcal{D}$ on which $M$ achieves low loss. The task of behavior removal is defined as modifying $M$ to create another model $M'$ that achieves high loss on $\mathcal{D}$, without a significant increase of loss on inputs outside $\mathcal{D}$.

To ablate $M$, first is necessary to choose at what level of granularity represent the model's computations, and write the graph $G$ that describes $M$ at that specific level (e.g. represent $M$ as a graph of attention heads and feed forward layers). Then the ablated edges in $G$ are determined by solving:

\begin{equation}
    \min_{W} \mathcal{L}(G_W, D_{\text{train}}) -  \alpha\mathcal{L}(G_W, \mathcal{D}) + \lambda(t)R(W)
\end{equation}

where $G_W$ is $M$ with a mask $W$ applied over the edges of $G$, $D_{\text{train}}$ is a set of train data, disjoint to $\mathcal{D}$, $R$ is a regularization function, $\alpha$ is a constant, and $\lambda(t)$ a regularization weight that increase over time. The mask $W$ assigns to each edge $e=(A,B)$ of $G$ a weight $w_e\in[0,1]$, such that node $B$ recieves the following combination of the original value $v_A$ and the ablated value $\mu_A$ from node $A$:

\begin{equation}
    w_ev_A + (1-w_e)\mu_A
\end{equation}

After the optimization is finished, the edges whose weight does not surpase a specified threshold are ablated.

If $\mathcal{D}$ is set to represent a biased behavior, and TEA is capable of effectively removing it from the model, then that would comfirm that $M$ computes the specified bias. Moreover, if combined with a method to analize the representations, it could help to identify what is the bias association being computed by $M'$, by comparing the inner representations before and after the ablation.\cite{geometry-of-truth}

\section{Example Generation}
\label{sec:examples}

\subsection{Counterfactual Examples}
\label{sec:examples:counterfactual}

In machine learning, a counterfactual explanation is an example that illustrates how a different input would result in a different output~\cite{NLP-posthoc-interpret-survey}. In NLP task, such as text classification or next token prediction, for instance, if an input text $X$ gets an output $y$ by the model, a counterfactual example would be any text $X'$, similar to $X$, that yields a different output $y'$~\cite{counterfactual-explanations,NLP-counterfactual-survey}. Though this definition is clear and widely used, it may be too narrow, as it leaves out the sense probability in the predictions.

\subsubsection{POLYJUICE}

The POLYJUICE~\cite{polyjuice} method employs a fine-tuned GPT-2~\cite{GPT-2} model to generate counterfactual examples. The model recieves an input text $X$ and a perturbation instruction, such as negation, insertion or deletion, and returns a modified version of $X$, following the given instruction.

\subsubsection{MiCE}

The Minimal Contrastive Editing (MiCE)~\cite{MiCE} method generate counterfactual examples via a masked language model, called editor model, that is trained to fill the masked spaces in an intput text with tokens, such that the resulting text would obtain a given prediction by some predictive model $M$. The editor model recieves the masked text and the target label as inputs. During training, the top $n_1\%$ tokens with highest gradient attribution~\cite{gradient-attribution}, towards the target prediction, are masked. To generate the counterfactuals, the percentage of masked tokens is varied between $0\%$ and $55\%$, using binary search to find the optimal percentage and beam search to keep track of the edits. The generation process stops once an edit changes the prediction to the target.

\subsubsection{GYC}

The GYC~\cite{GYC} method generates $k$ counterfactual examples, for a given input text $X$ and a condition $C$, through controlled text generation, without requiring to train or fine-tune a model. The method consists in modelating the distribution $p(\tilde{y}|X, C)$, where the condition $C$ can be any restriction over the text, such as a class label.

Let $LM$ be a language model transformer; $LM$ generates a token $y_t$ conditioned on the past tokens $y_{<t}=\{y_i\}^{t-1}_{i=0}$ as follows:

\begin{equation}
    \begin{split}
        o_{t}, H_{t} &= \text{LM}(y_{t-1}, H_{t-1})\\
        y_t &\sim \text{Categorical}(o_t)
    \end{split}
\end{equation}

where $H_{t-1}$ is the history matrix that captures the dependency of $y_{t-1}$ on past tokens and $o_t$ are the logits to sample $y_t$ from a categorical distribution. To generate text conditioned on $C$, $H$ is perturbed two times, first to create $\tilde{H}_t$ which enforces the reconstruction of $X$, and next to create $\hat{H}_t$ which enforces the condition. To learn the perturbations a linear combination of three loss functions is employed, one for reconstruction and one enforce the condition, plus another one to ensure diversity. The reconstruction loss maximizes the log probability of the input text, the condition loss maximizes a score assosiated to the condition and the diversity loss maximizes the entropy of the generated logits.


\subsection{Activation Maximization}
\label{sec:examples:activation_max}



\subsubsection{Preferred Input Synthesis}

The Preferred Input Synthesis (PIS)~\cite{synthetisizing-Nguyen} method generates inputs that maximize the activation of a target neuron (including the output logits), via optimization at the level of the latent space of a generative model. Given a predictive model $M:X\to Y$ and a generative model $G:Z\to X$, denoting by $M_h$ the activation of the target neuron $h$, PIS generate samples by solving:

\begin{equation}
    \argmax_{z\in Z}M_h(G(z)) - \lambda\|z\|
\end{equation}

where $\lambda$ is a regularization term and the optimization is performed with gradient descend. PIS was originally formulated for image generation, but can easily be re-adapted for text.


\subsubsection{Momentum Evolutionary Input Optimization}

The Evolutionary Input Optimization (MEIO)~\cite{synthetisizing-Barbalau} method is a generic framework for Activation Maximization that propose a model-agnostic approch, with a zero-order optimization on the latent space of a generative model. Given a predictive model $M:X\to Y$, a generative model $G:Z\to X$, and a target prediction $y\in Y$, MEIO generate samples by solving:

\begin{equation}
    \min_{z\in Z}\mathcal{L}(M(G(z)), y)
\end{equation}

where $\mathcal{L}$ is a loss function. The optimization is performed via an evolutionary strategy with momentum updates, that involves iteratively updating a set of candidate solutions, by adding to them an idenpdently sampled gaussian noise, which is combined with noised added in the previous itariton (the momentum).


\section{Conclusion}
\label{sec:conclusion}

\begin{figure}
    \centering
    \includegraphics[width=\textwidth]{Imgs/bias_diagram2.drawio.pdf}
    \caption{Diagram of the relation between Social Bias, Prediction Bias, Exploratory Analysis, Confirmatory Analysis and the concepts defined in the literature of Concept Erasure and Model Ablation.}
    \label{fig:bias-diagram2}
\end{figure}

In order to effectively prevent or remove unintended bias in a predictive model, it is necessary to know how this bias is being computed in the model, not only its concequences or causes. We denominate these computations as Prediction Bias, and define Prediction Bias Analysis as the analysis that seeks to identify or prove associations between variables encoded in a predictive model. Prediction Bias Analysis can be oriented to confirm bias in a known variable, or to explore for bias in unknown previuosly variables. In this survey, we review five families of explainability methods and how they can be repurposed to perform Prediction Bias Analysis. These methods are divided into Examination methods, that examinate components of the prediction process, and Example Generation methods, that generate input samples to produce a specific response on the model.

Examination methods can effectively identify and measure the effect of variables in the input over the responses of a predictive model. However, when employed for Exploratory Analysis, the associations they find are difficult to comprehend. In contrast, Example Generation methods can be applied to elucidate associations between input features and the model's predictions in a format closer to human understanding. Nevertheless, the examples need further analysis to identify or measure Prediction Bias. In this sense, Confirmatory and Exploratory Analysis, and Examination and Example Generation methods, are complements of each others. Confirmatory Analysis provides the means to assess the effects of the variables found by Exploratory Analysis; and Example Generation can provide further interpretability to Examination methods.

Even though the majority of the methods reviewed in this survey were not designed for Prediction Bias Analysis, they are founded on a common theoretical framework and have compatible objectives. Consequently, they can be repurposed to build a better framework for Prediction Bias Analysis. Concept Erasure defines a concept in the vector representations as a subspace $\mathcal{B}$ that renders the response variable $\mathcal{C}$ sensitive to them. Model Ablation defines a behaivor as a set of inputs $\mathcal{D}$ where a model achieve low values of a given loss function. The idea of concept correlates with the variables encoded in the computations of a model, and the behaivor correspond to a particular effect on the model's predictions. Thus, Prediction Bias Analysis can be defined as the analysis that seeks to identify a concept $\mathcal{B}$ that causes a behaivor $\mathcal{D}$ or measure the effect of a concept $\mathcal{C}$ over a behaivor $\mathcal{D}$. With this definition, the Prediction Bias would be the causal relation between $(\mathcal{B},\mathcal{C})$ and $\mathcal{D}$ (figure \ref{fig:bias-diagram2}).





%
% ---- Bibliography ----
%
% BibTeX users should specify bibliography style 'splncs04'.
% References will then be sorted and formatted in the correct style.
%
\bibliographystyle{splncs04}
\bibliography{References}
%

\end{document}
