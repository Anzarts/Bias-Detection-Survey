\section{Conclusion and Research Oportunities}
\label{sec:conclusion}

\begin{figure}
    \centering
    \includegraphics[width=\textwidth]{Imgs/bias_diagram2.drawio.pdf}
    \caption{Diagram of the relation between Social Bias, Prediction Bias, Exploratory Analysis, Confirmatory Analysis and the concepts defined in the literature of Concept Erasure and Model Ablation.}
    \label{fig:bias-diagram2}
\end{figure}

In order to effectively prevent or remove unintended bias in a predictive model, it is necessary to know how this bias is being computed in the model, not only its concequences or causes. We denominate these computations as Prediction Bias, and define Prediction Bias Analysis as the analysis that seeks to identify or prove associations between variables encoded in a predictive model. Prediction Bias Analysis can be oriented to confirm bias in a known variable, or to explore for bias in unknown previuosly variables. In this survey, we review five families of explainability methods and how they can be repurposed to perform Prediction Bias Analysis. These methods are divided into Examination methods, that examinate components of the prediction process, and Example Generation methods, that generate input samples to produce a specific response on the model.

Examination methods can effectively identify and measure the effect of variables in the input over the responses of a predictive model. However, when employed for Exploratory Analysis, the associations they find are difficult to comprehend. In contrast, Example Generation methods can be applied to elucidate associations between input features and the model's predictions in a format closer to human understanding. Nevertheless, the examples need further analysis to identify or measure Prediction Bias. In this sense, Confirmatory and Exploratory Analysis, and Examination and Example Generation methods, are complements of each others. Confirmatory Analysis provides the means to assess the effects of the variables found by Exploratory Analysis; and Example Generation can provide further interpretability to Examination methods.

Even though the majority of the methods reviewed in this survey were not designed for Prediction Bias Analysis, they are founded on a common theoretical framework and have compatible objectives. Consequently, they can be repurposed to build a better framework for Prediction Bias Analysis. Concept Erasure defines a concept in the vector representations as a subspace $\mathcal{B}$ that renders the response variable $\mathcal{C}$ sensitive to them. Model Ablation defines a behaivor as a set of inputs $\mathcal{D}$ where a model achieve low values of a given loss function. The idea of concept correlates with the variables encoded in the computations of a model, and the behaivor correspond to a particular effect on the model's predictions. Thus, Prediction Bias Analysis can be defined as the analysis that seeks to identify a concept $\mathcal{B}$ that causes a behaivor $\mathcal{D}$ or measure the effect of a concept $\mathcal{C}$ over a behaivor $\mathcal{D}$. With this definition, the Prediction Bias would be the causal relation between $(\mathcal{B},\mathcal{C})$ and $\mathcal{D}$ (figure \ref{fig:bias-diagram2}).

\subsection{Research Oportunities}
\label{sec:conclusion:RO}

To finalize this survey, we identify the following research oportunities for Prediction Bias Analysis:

\begin{itemize}
    \item \textbf{Develope a formal framework:} A well defined mathematical framework for prediction bias is necessary for a better understanding of the concept, to develope more robust methods, and for a better integretion with other existing paradigms and methodologies.
    \item \textbf{Develope specific methods:} In this work we reviewed methods that were not originally intended for Prediction Bias Analysis and discussed how they could be repurposed for this objective. These repurposed methods can be used as starting point for the development of mthods specifically designed to identify prediction bias. 
    \item \textbf{More exploratory analisys:} Exploratory and Confirmatory Analysis are completary. Confirmatory Analysis can provide results more robust and interpretable; however, it requires prior knowledge regarding the bias. This knowledge is commonly adquired after using the model and witnessing the consequences of the bias. Exploratory Analysis offers the advantage of adquiring this knowledge in advance. Further improvement of these techniques has the potential of enhancing the capacity of Confirmatory Analysis and providing a more effective prevention of bias.
\end{itemize}z
