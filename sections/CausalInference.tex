\subsection{Direct and Indirect Effects}
\label{sec:backgroun:indirect-effect}

In our hiring model example (\textsc{Ex1} and \textsc{Ex2}), the model express a gender bias, but the actual association encoded in the model is between the length of sentences and variety of words, which correlate with gender, and the qualification for the job. In fact, the model does not even recieve any information regardign gender. In this toy scenario, gender has no direct effect, but it has a significant indirect effect.

\subsubsection{Direct Effect}
\label{sec:backgroun:indirect-effect:DE}

The effect that a variable has over an outcome can be decomposed as the sum of its direct effect and its indirect effect. The direct effect is a quantification of the influence that a variable has on an outcome, that is not mediated by other variables. Let $X$ be the varieble whose effect we seek to assess, $Y$ the response variable, and $Z$ the set of all the intermediate variables between $X$ and $Y$. The direct effect of $X$ over $Y$ measures the sesivity of $Y$ to changes in $X$, while $Z$ is held fixed~\cite{indirect-effect}. Formaly, the direct effect of an event $X=x$ is given by

\begin{equation}
    DE(x, x^*;Y)=Y_{xZ_{x^*}} - Y_{x^*}
\end{equation}

where $x^*$ is a reference value for $X$ and the notation $Y_x$ is used to represent the value $Y$ would attain when $X$ is set to be $x$.

\subsubsection{Indirect Effect}
\label{sec:backgroun:indirect-effect:IE}

In contrast to the direct effect, the indirect effect of an event $X=x$ quantify the sensitivity of $Y$, to changes in the mediators $Z$ induced by $X=x$. This is equivalent to measure the change in $Y$ when $Z$ is set to the value it would attain under $X=x$, while $X$ is held fixed at the reference value~\cite{indirect-effect}. Formaly, the indirect effect is given by

\begin{equation}
    IE(x, x^*;Y)=Y_{x^*Z_{x}} - Y_{x^*}
\end{equation}

The indirect effect is an important factor to consider when analysing bias that is often neglected. In the hiring model example (\textsc{Ex1}), the idea that the model has a bias realated to $X$ (gender) is rejected at first, because is obvious that $X$ has no direct on $Y$ (the output). The inderect effect, and subsequently the bias, is in \textsc{Ex2} probed after finding, by chance, the mediator $Z$ (length of sentences and variety of words) between $X$ and $Y$. In \textsc{Ex1} is evident that $X$ has no direct effect, as it is not part of the input, but there can be similar escenarios where $X$ is an attribute of the input.

Consider the alternative version of the hiring example, \textsc{Ex3}, where the gender, $X$, is included in the input. In an attemp to debias the model, $X$ is modified in the training data, however $Z$ remains in its original value. As result, the model learns the same association between $Z$ and $Y$. The direct effect of $X$ on $Y$ is reduced, but the bias remains intact. In order to achieve an effective bias mitigation, it is imperative to ensure that the bias evaluation method is assessing the total effect and not only the direct effect.

If, as in the example, bias is dominated by an indirect effect, it may be necessary a characterization of the mediator variables $Z$ to effectively debias the model. However, in general, $Z$ is an unknown variable, and can correspond to a feature that is not explicity represented in the data. We denominate as confirmatory analysis to verify or measure the effect of $X$ over $Y$, and as exploratory analysis to to verify or measure the effect of a mediator $Z$, between $X$ and $Y$. In other words, confirmatory methods detect an association that involves a target variable, and exploratory methods can detect an association that involves a previously unknown variable.

%While $X$ can be directly observed and intervented, $Z$ must be found.

The objective of Prediction Bias Analysis to indentify association between variables encoded in a predictive model. That is equivalent to measure (or at least confirm the existence of) the total effect, i.e. the sum of direct and indirect effect, of a feature $X$ of the input over the outcome $Y$. In this context, the observed or target variable $X$ can be an explicit attribute of the input, like the declared gender in the example (\textsc{Ex3}), or it can be an external factor to the data, like the gender of the applicants in the examples. We will use the notation $X_D$ for variables of the former scenario and $X_S$ for variables of the later.


